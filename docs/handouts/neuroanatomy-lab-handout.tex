\documentclass[]{article}
\usepackage{lmodern}
\usepackage{amssymb,amsmath}
\usepackage{ifxetex,ifluatex}
\usepackage{fixltx2e} % provides \textsubscript
\ifnum 0\ifxetex 1\fi\ifluatex 1\fi=0 % if pdftex
  \usepackage[T1]{fontenc}
  \usepackage[utf8]{inputenc}
\else % if luatex or xelatex
  \ifxetex
    \usepackage{mathspec}
  \else
    \usepackage{fontspec}
  \fi
  \defaultfontfeatures{Ligatures=TeX,Scale=MatchLowercase}
\fi
% use upquote if available, for straight quotes in verbatim environments
\IfFileExists{upquote.sty}{\usepackage{upquote}}{}
% use microtype if available
\IfFileExists{microtype.sty}{%
\usepackage{microtype}
\UseMicrotypeSet[protrusion]{basicmath} % disable protrusion for tt fonts
}{}
\usepackage[margin=1in]{geometry}
\usepackage{hyperref}
\hypersetup{unicode=true,
            pdftitle={neuroanatomy-lab},
            pdfauthor={Rick Gilmore},
            pdfborder={0 0 0},
            breaklinks=true}
\urlstyle{same}  % don't use monospace font for urls
\usepackage{longtable,booktabs}
\usepackage{graphicx,grffile}
\makeatletter
\def\maxwidth{\ifdim\Gin@nat@width>\linewidth\linewidth\else\Gin@nat@width\fi}
\def\maxheight{\ifdim\Gin@nat@height>\textheight\textheight\else\Gin@nat@height\fi}
\makeatother
% Scale images if necessary, so that they will not overflow the page
% margins by default, and it is still possible to overwrite the defaults
% using explicit options in \includegraphics[width, height, ...]{}
\setkeys{Gin}{width=\maxwidth,height=\maxheight,keepaspectratio}
\IfFileExists{parskip.sty}{%
\usepackage{parskip}
}{% else
\setlength{\parindent}{0pt}
\setlength{\parskip}{6pt plus 2pt minus 1pt}
}
\setlength{\emergencystretch}{3em}  % prevent overfull lines
\providecommand{\tightlist}{%
  \setlength{\itemsep}{0pt}\setlength{\parskip}{0pt}}
\setcounter{secnumdepth}{0}
% Redefines (sub)paragraphs to behave more like sections
\ifx\paragraph\undefined\else
\let\oldparagraph\paragraph
\renewcommand{\paragraph}[1]{\oldparagraph{#1}\mbox{}}
\fi
\ifx\subparagraph\undefined\else
\let\oldsubparagraph\subparagraph
\renewcommand{\subparagraph}[1]{\oldsubparagraph{#1}\mbox{}}
\fi

%%% Use protect on footnotes to avoid problems with footnotes in titles
\let\rmarkdownfootnote\footnote%
\def\footnote{\protect\rmarkdownfootnote}

%%% Change title format to be more compact
\usepackage{titling}

% Create subtitle command for use in maketitle
\providecommand{\subtitle}[1]{
  \posttitle{
    \begin{center}\large#1\end{center}
    }
}

\setlength{\droptitle}{-2em}

  \title{neuroanatomy-lab}
    \pretitle{\vspace{\droptitle}\centering\huge}
  \posttitle{\par}
    \author{Rick Gilmore}
    \preauthor{\centering\large\emph}
  \postauthor{\par}
      \predate{\centering\large\emph}
  \postdate{\par}
    \date{2019-09-17}


\begin{document}
\maketitle

\hypertarget{directional-terms}{%
\subsection{Directional terms}\label{directional-terms}}

\begin{longtable}[]{@{}llll@{}}
\toprule
Direction & Brain Model & Specimen & Atlas\tabularnewline
\midrule
\endhead
Rostral/caudal & & &\tabularnewline
Anterior/posterior & & &\tabularnewline
Medial/lateral & & &\tabularnewline
Dorsal/ventral & & &\tabularnewline
Superior/inferior & & &\tabularnewline
\bottomrule
\end{longtable}

\begin{center}\rule{0.5\linewidth}{\linethickness}\end{center}

\hypertarget{planes-of-section}{%
\subsection{Planes of Section}\label{planes-of-section}}

\begin{longtable}[]{@{}llll@{}}
\toprule
Plane & Brain Model & Specimen & Atlas\tabularnewline
\midrule
\endhead
\href{https://en.wikipedia.org/wiki/Sagittal_plane}{Sagittal} & &
&\tabularnewline
\href{https://en.wikipedia.org/wiki/Coronal_plane}{Coronal/frontal} & &
&\tabularnewline
\href{https://en.wikipedia.org/wiki/Transverse_plane}{Axial/horizontal/transverse}
& & &\tabularnewline
\bottomrule
\end{longtable}

\begin{center}\rule{0.5\linewidth}{\linethickness}\end{center}

\hypertarget{brain-structure}{%
\subsection{Brain Structure}\label{brain-structure}}

\begin{longtable}[]{@{}llll@{}}
\toprule
Structure & Brain Model & Specimen & Atlas\tabularnewline
\midrule
\endhead
\href{https://en.wikipedia.org/wiki/Prosencephalon}{Forebrain} & &
&\tabularnewline
\href{https://en.wikipedia.org/wiki/Midbrain}{Midbrain} & &
&\tabularnewline
\href{https://en.wikipedia.org/wiki/Rhombencephalon}{Hindbrain} & &
&\tabularnewline
\bottomrule
\end{longtable}

\begin{center}\rule{0.5\linewidth}{\linethickness}\end{center}

\hypertarget{surface-of-cerebral-cortex}{%
\subsection{Surface of Cerebral
Cortex}\label{surface-of-cerebral-cortex}}

\begin{longtable}[]{@{}llll@{}}
\toprule
Structure & Brain Model & Specimen & Atlas\tabularnewline
\midrule
\endhead
\href{https://en.wikipedia.org/wiki/Central_sulcus}{Central sulcus} & &
&\tabularnewline
\href{https://en.wikipedia.org/wiki/Lateral_sulcus}{Lateral
sulcus/fissure} & & &\tabularnewline
\href{https://en.wikipedia.org/wiki/Medial_longitudinal_fissure}{Longitudinal
fissure} & & &\tabularnewline
\href{https://en.wikipedia.org/wiki/Cerebral_hemisphere}{Cerebral
hemispheres} & & &\tabularnewline
\href{https://en.wikipedia.org/wiki/Parietal_lobe}{Parietal lobe} & &
&\tabularnewline
\href{https://en.wikipedia.org/wiki/Frontal_lobe}{Frontal lobe} & &
&\tabularnewline
\href{https://en.wikipedia.org/wiki/Insular_cortex}{Insular cortex} & &
&\tabularnewline
\bottomrule
\end{longtable}

\begin{center}\rule{0.5\linewidth}{\linethickness}\end{center}

\hypertarget{surface-of-cerebral-cortex-1}{%
\subsection{Surface of Cerebral
Cortex}\label{surface-of-cerebral-cortex-1}}

\begin{longtable}[]{@{}llll@{}}
\toprule
Structure & Brain Model & Specimen & Atlas\tabularnewline
\midrule
\endhead
\href{https://en.wikipedia.org/wiki/Temporal_lobe}{Temporal lobe} & &
&\tabularnewline
\href{https://en.wikipedia.org/wiki/Occipital_lobe}{Occipital lobe} & &
&\tabularnewline
\bottomrule
\end{longtable}

\hypertarget{surface-of-cerebral-cortex-2}{%
\subsection{Surface of Cerebral
Cortex}\label{surface-of-cerebral-cortex-2}}

\begin{longtable}[]{@{}llll@{}}
\toprule
Structure & Brain Model & Specimen & Atlas\tabularnewline
\midrule
\endhead
\href{https://en.wikipedia.org/wiki/Precentral_gyrus}{Precentral gyrus}
& & &\tabularnewline
\href{https://en.wikipedia.org/wiki/Postcentral_gyrus}{Postcentral
gyrus} & & &\tabularnewline
\href{https://en.wikipedia.org/wiki/Superior_temporal_gyrus}{Superior
temporal gyrus} & & &\tabularnewline
\bottomrule
\end{longtable}

\hypertarget{fiber-tracts}{%
\subsection{Fiber tracts}\label{fiber-tracts}}

\begin{longtable}[]{@{}llll@{}}
\toprule
Structure & Brain Model & Specimen & Atlas\tabularnewline
\midrule
\endhead
\href{https://en.wikipedia.org/wiki/Corpus_callosum}{Corpus callosum} &
& &\tabularnewline
\href{https://en.wikipedia.org/wiki/Anterior_commissure}{Anterior
commissure} & & &\tabularnewline
\href{https://en.wikipedia.org/wiki/Posterior_commissure}{Posterior
commissure} & & &\tabularnewline
\href{https://en.wikipedia.org/wiki/Olfactory_nerve}{Olfactory nerve,
Ist} & & &\tabularnewline
\href{https://en.wikipedia.org/wiki/Optic_nerve}{Optic nerve/tract,
IInd} & & &\tabularnewline
\href{https://en.wikipedia.org/wiki/Optic_chiasm}{Optic chiasm} & &
&\tabularnewline
\href{https://en.wikipedia.org/wiki/Fornix_(neuroanatomy}{Fornix} & &
&\tabularnewline
\bottomrule
\end{longtable}

\hypertarget{subcortical-structures}{%
\subsection{Subcortical structures}\label{subcortical-structures}}

\begin{longtable}[]{@{}llll@{}}
\toprule
Structure & Brain Model & Specimen & Atlas\tabularnewline
\midrule
\endhead
\href{https://en.wikipedia.org/wiki/Caudate_nucleus}{Caudate nucleus} &
& &\tabularnewline
\href{https://en.wikipedia.org/wiki/Putamen}{Putamen} & &
&\tabularnewline
\href{https://en.wikipedia.org/wiki/Globus_pallidus}{Globus Pallidus} &
& &\tabularnewline
\href{https://en.wikipedia.org/wiki/Thalamus}{Thalamus} & &
&\tabularnewline
\bottomrule
\end{longtable}

\hypertarget{subcortical-structures-1}{%
\subsection{Subcortical structures}\label{subcortical-structures-1}}

\begin{longtable}[]{@{}llll@{}}
\toprule
Structure & Brain Model & Specimen & Atlas\tabularnewline
\midrule
\endhead
\href{https://en.wikipedia.org/wiki/Hypothalamus}{Hypothalamus} & &
&\tabularnewline
\href{https://en.wikipedia.org/wiki/Pituitary_gland}{Pituitary gland} &
& &\tabularnewline
\href{https://en.wikipedia.org/wiki/Hippocampus}{Hippocampus} & &
&\tabularnewline
\href{https://en.wikipedia.org/wiki/Amygdala}{Amygdala} & &
&\tabularnewline
\bottomrule
\end{longtable}

\begin{center}\rule{0.5\linewidth}{\linethickness}\end{center}

\hypertarget{midbrain}{%
\subsection{Midbrain}\label{midbrain}}

\begin{longtable}[]{@{}llll@{}}
\toprule
Structure & Brain Model & Specimen & Atlas\tabularnewline
\midrule
\endhead
\href{https://en.wikipedia.org/wiki/Midbrain_tectum}{Tectum} & &
&\tabularnewline
\href{https://en.wikipedia.org/wiki/Superior_colliculus}{Superior
colliculus} & & &\tabularnewline
\href{https://en.wikipedia.org/wiki/Inferior_colliculus}{Inferior
colliculus} & & &\tabularnewline
\href{https://en.wikipedia.org/wiki/Tegmentum}{Tegmentum} & &
&\tabularnewline
\href{https://en.wikipedia.org/wiki/Substantia_nigra}{Substantia nigra}
& & &\tabularnewline
\href{https://en.wikipedia.org/wiki/Ventral_tegmental_area}{Ventral
tegmental area} & & &\tabularnewline
\href{https://en.wikipedia.org/wiki/Locus_coeruleus}{Locus coeruleus} &
& &\tabularnewline
\bottomrule
\end{longtable}

\hypertarget{hindbrain}{%
\subsection{Hindbrain}\label{hindbrain}}

\begin{longtable}[]{@{}llll@{}}
\toprule
Structure & Brain Model & Specimen & Atlas\tabularnewline
\midrule
\endhead
\href{https://en.wikipedia.org/wiki/Cerebellum}{Cerebellum} & &
&\tabularnewline
\href{https://en.wikipedia.org/wiki/Pons}{Pons} & & &\tabularnewline
\href{https://en.wikipedia.org/wiki/Medulla_oblongata}{Medulla
Oblongata} & & &\tabularnewline
\href{https://en.wikipedia.org/wiki/Spinal_cord}{Spinal cord} & &
&\tabularnewline
\bottomrule
\end{longtable}

\hypertarget{ventricles}{%
\subsection{Ventricles}\label{ventricles}}

\begin{longtable}[]{@{}llll@{}}
\toprule
Structure & Brain Model & Specimen & Atlas\tabularnewline
\midrule
\endhead
\href{https://en.wikipedia.org/wiki/Lateral_ventricles}{Lateral
ventricles} & & &\tabularnewline
\href{https://en.wikipedia.org/wiki/Third_ventricle}{3rd ventricle} & &
&\tabularnewline
\href{https://en.wikipedia.org/wiki/Cerebral_aqueduct}{Cerebral
aqueduct} & & &\tabularnewline
\href{https://en.wikipedia.org/wiki/Fourth_ventricle}{4th ventricle} & &
&\tabularnewline
\bottomrule
\end{longtable}


\end{document}
